
%% bare_jrnl.tex
%% V1.4b
%% 2015/08/26
%% by Michael Shell
%% see http://www.michaelshell.org/
%% for current contact information.
%%
%% This is a skeleton file demonstrating the use of IEEEtran.cls
%% (requires IEEEtran.cls version 1.8b or later) with an IEEE
%% journal paper.
%%
%% Support sites:
%% http://www.michaelshell.org/tex/ieeetran/
%% http://www.ctan.org/pkg/ieeetran
%% and
%% http://www.ieee.org/

%%*************************************************************************
%% Legal Notice:
%% This code is offered as-is without any warranty either expressed or
%% implied; without even the implied warranty of MERCHANTABILITY or
%% FITNESS FOR A PARTICULAR PURPOSE! 
%% User assumes all risk.
%% In no event shall the IEEE or any contributor to this code be liable for
%% any damages or losses, including, but not limited to, incidental,
%% consequential, or any other damages, resulting from the use or misuse
%% of any information contained here.
%%
%% All comments are the opinions of their respective authors and are not
%% necessarily endorsed by the IEEE.
%%
%% This work is distributed under the LaTeX Project Public License (LPPL)
%% ( http://www.latex-project.org/ ) version 1.3, and may be freely used,
%% distributed and modified. A copy of the LPPL, version 1.3, is included
%% in the base LaTeX documentation of all distributions of LaTeX released
%% 2003/12/01 or later.
%% Retain all contribution notices and credits.
%% ** Modified files should be clearly indicated as such, including  **
%% ** renaming them and changing author support contact information. **
%%*************************************************************************


% *** Authors should verify (and, if needed, correct) their LaTeX system  ***
% *** with the testflow diagnostic prior to trusting their LaTeX platform ***
% *** with production work. The IEEE's font choices and paper sizes can   ***
% *** trigger bugs that do not appear when using other class files.       ***                          ***
% The testflow support page is at:
% http://www.michaelshell.org/tex/testflow/

% formatting instructions
% https://ras.papercept.net/conferences/support/files/IEEEtran_HOWTO.pdf

% from here
%https://journals.ieeeauthorcenter.ieee.org/create-your-ieee-journal-article/authoring-tools-and-templates/ieee-article-templates/templates-for-transactions/

\documentclass[journal]{IEEEtran}

\usepackage{cite}
\usepackage{amsmath}
\usepackage{url}
\usepackage{multirow}

% *** GRAPHICS RELATED PACKAGES ***
%
\ifCLASSINFOpdf
  \usepackage[pdftex]{graphicx}
  % declare the path(s) where your graphic files are
\graphicspath{{../images/}}
  % and their extensions so you won't have to specify these with
  % every instance of \includegraphics
  % \DeclareGraphicsExtensions{.pdf,.jpeg,.png}
\else
  % or other class option (dvipsone, dvipdf, if not using dvips). graphicx
  % will default to the driver specified in the system graphics.cfg if no
  % driver is specified.
  % \usepackage[dvips]{graphicx}
  % declare the path(s) where your graphic files are
  % \graphicspath{{../eps/}}
  % and their extensions so you won't have to specify these with
  % every instance of \includegraphics
  % \DeclareGraphicsExtensions{.eps}
\fi

\begin{document}

\title{Big Data Solution for\\ Stock Prices and Tweet Collections}

\author{Paul~Adams, paula@smu.edu,
        Rikel~Djoko, rdjoko@smu.edu,
        and Stuart Miller, stuart@smu.edu}% <-this % stops a space

% The paper headers
\markboth{MSDS7330}{MSDS7330}
% The only time the second header will appear is for the odd numbered pages
% after the title page when using the twoside option.

% make the title area
\maketitle

% This should be about 250 words
\begin{abstract}
The abstract goes here.
\end{abstract}


% For peer review papers, you can put extra information on the cover
% page as needed:
% \ifCLASSOPTIONpeerreview
% \begin{center} \bfseries EDICS Category: 3-BBND \end{center}
% \fi
%
% For peerreview papers, this IEEEtran command inserts a page break and
% creates the second title. It will be ignored for other modes.
\IEEEpeerreviewmaketitle



\section{Introduction}
% The very first letter is a 2 line initial drop letter followed
% by the rest of the first word in caps.
% 
% form to use if the first word consists of a single letter:
% \IEEEPARstart{A}{demo} file is ....
% 
% form to use if you need the single drop letter followed by
% normal text (unknown if ever used by the IEEE):
% \IEEEPARstart{A}{}demo file is ....
% 
% Some journals put the first two words in caps:
% \IEEEPARstart{T}{his demo} file is ....
% 
% Here we have the typical use of a "T" for an initial drop letter
% and "HIS" in caps to complete the first word.
\IEEEPARstart{D}{ata} in the twenty-first century is exploding in volumes at an exponential rate; every additional source of data that can act as a medium for data communication can obtain useful information, which, with modern technology, can be structured and stored, accessible to any who have the skills and need to make use of it. 


\subsection{Apache Hadoop Ecosystem}

something

\subsection{Data Selection and Processing}

Stock and Twitter

\subsection{Big Data System Implementation}

Implementation with AWS




\section{Data}

This study investigates data warehouse models for housing stock price data and semi-structured alternative data. 
The stock price data was.
Twitter was chosen as the primary alternative data source because of ease-of-access the Twitter API and the large volume of available data.

\begin{table}
	\renewcommand{\arraystretch}{1.3}
	\caption{Data Features}
	\label{DataFeatures}
	\centering
	\begin{tabular}{c|l}
		\hline
		Source       & Features\\
		\hline
		\hline
		Stock Daily  & Prices: High, Low, Open, Close\\
		\hline
		\multirow{9}{*}{Stock Intraday} &  Prices: High, Low, Open, Close \\
		&  High Bollinger bands\\
		&  Mid Bollinger bands\\
		&  Low Bollinger bands\\ 
		&  Nominal moving average\\
		&  Historical moving average\\
		&  Signal moving average\\ 
		&  Exponential moving average\\
		&  Stochastic Slow K\\
		&  Stochastic Slow D\\
		\hline
		Tweets       & Text, URLs, Hashtags, Mentions, Users\\
		\hline
	\end{tabular}
\end{table}

\subsection{Stock Data}

The stock price data was collected through an API provided by Alpha Vantage.
This API provided access intraday stock prices, intraday price features, and daily prices.
Intraday stock data contained 35 features sampled at 15 minute intervals. 
The intraday features were from the following categories Bollinger bands, stochastic oscillators, moving averages, and exponential moving averages.
Approximately 1 millions rows of data were collected, which occurred from Oct. 04, 2019 to Oct. 24, 2019.
During the collection process, the data were recorded in files organized by day and category.
At the end of the collection process, the files for each category were combined and pushed to the HDFS datalake.


\subsection{Twitter Data}

sssss

\section{Data Warehouse Development}

This stock data is natively structured in a tabular form and naturally keyed by the combination of timestamp, stock symbol, and stock market.

\subsection{Normalized Schema}

%\begin{figure}
%	\centering
%	\includegraphics[width=2.5in]{NormalizedSchema.png}
%	\caption{Normalized Data Warehouse Schema}
%	\label{fig_sim}
%\end{figure}

Discuss normalizing the data - 3NF

\subsection{Optimized Schema}

Discuss the difference for optimization

\section{Technologies}

Rikel

\section{Results}



Maybe a table comparing read times for the different options

\begin{itemize}
	\item S3
	\item HDFS
	\item HDFS mapreduce settings?
\end{itemize}

\section{Analysis}

Analysis of the results


\section{Conclusion}

The conclusion goes here.

% Can use something like this to put references on a page
% by themselves when using endfloat and the captionsoff option.
\ifCLASSOPTIONcaptionsoff
  \newpage
\fi

\begin{thebibliography}{1}

\bibitem{BuildingtheDWCH11}
W. H. Inmon, "Unstructured Data and the Data Warehouse," in 
  \emph{Building the Data Warehouse},
  4th ed. Hoboken: Wiley, 2005, ch. x, sec. x.
  Accessed on Nov. 6, 2019 [Online]. 
  Available: \\ https://learning.oreilly.com/library/view/building-the-data/9780764599446

\bibitem{WarehouseDesignApproaches}
I. Moalla, A. Nabli, L. Bouzguendam and M. Hammami,
 "Data warehouse design approaches from social media: review and comparison,"
 Social Network Analysis and Mining., Vol. 7, no. 1, pp. 1-14, Jan. 2017.
 Accessed on: Nov. 6, 2019 [Online]. Available doi: 10.1007/s13278-017-0423-8

\end{thebibliography}


% that's all folks
\end{document}


