
%% bare_jrnl.tex
%% V1.4b
%% 2015/08/26
%% by Michael Shell
%% see http://www.michaelshell.org/
%% for current contact information.
%%
%% This is a skeleton file demonstrating the use of IEEEtran.cls
%% (requires IEEEtran.cls version 1.8b or later) with an IEEE
%% journal paper.
%%
%% Support sites:
%% http://www.michaelshell.org/tex/ieeetran/
%% http://www.ctan.org/pkg/ieeetran
%% and
%% http://www.ieee.org/

%%*************************************************************************
%% Legal Notice:
%% This code is offered as-is without any warranty either expressed or
%% implied; without even the implied warranty of MERCHANTABILITY or
%% FITNESS FOR A PARTICULAR PURPOSE! 
%% User assumes all risk.
%% In no event shall the IEEE or any contributor to this code be liable for
%% any damages or losses, including, but not limited to, incidental,
%% consequential, or any other damages, resulting from the use or misuse
%% of any information contained here.
%%
%% All comments are the opinions of their respective authors and are not
%% necessarily endorsed by the IEEE.
%%
%% This work is distributed under the LaTeX Project Public License (LPPL)
%% ( http://www.latex-project.org/ ) version 1.3, and may be freely used,
%% distributed and modified. A copy of the LPPL, version 1.3, is included
%% in the base LaTeX documentation of all distributions of LaTeX released
%% 2003/12/01 or later.
%% Retain all contribution notices and credits.
%% ** Modified files should be clearly indicated as such, including  **
%% ** renaming them and changing author support contact information. **
%%*************************************************************************


% *** Authors should verify (and, if needed, correct) their LaTeX system  ***
% *** with the testflow diagnostic prior to trusting their LaTeX platform ***
% *** with production work. The IEEE's font choices and paper sizes can   ***
% *** trigger bugs that do not appear when using other class files.       ***                          ***
% The testflow support page is at:
% http://www.michaelshell.org/tex/testflow/

% formatting instructions
% https://ras.papercept.net/conferences/support/files/IEEEtran_HOWTO.pdf

% from here
%https://journals.ieeeauthorcenter.ieee.org/create-your-ieee-journal-article/authoring-tools-and-templates/ieee-article-templates/templates-for-transactions/

\documentclass[journal]{IEEEtran}

\usepackage{cite}
\usepackage{amsmath}
\usepackage{url}
\usepackage{multirow}

% *** GRAPHICS RELATED PACKAGES ***
%
\ifCLASSINFOpdf
  \usepackage[pdftex]{graphicx}
  % declare the path(s) where your graphic files are
\graphicspath{{../images/}}
  % and their extensions so you won't have to specify these with
  % every instance of \includegraphics
  % \DeclareGraphicsExtensions{.pdf,.jpeg,.png}
\else
  % or other class option (dvipsone, dvipdf, if not using dvips). graphicx
  % will default to the driver specified in the system graphics.cfg if no
  % driver is specified.
  % \usepackage[dvips]{graphicx}
  % declare the path(s) where your graphic files are
  % \graphicspath{{../eps/}}
  % and their extensions so you won't have to specify these with
  % every instance of \includegraphics
  % \DeclareGraphicsExtensions{.eps}
\fi

\begin{document}

\title{Big Data Solution for\\ Stock Prices and Tweet Collections}

\author{Paul~Adams, paula@smu.edu,
        Rikel~Djoko, rdjoko@smu.edu,
        and Stuart Miller, stuart@smu.edu}% <-this % stops a space

% The paper headers
\markboth{MSDS7330}{MSDS7330}
% The only time the second header will appear is for the odd numbered pages
% after the title page when using the twoside option.

% make the title area
\maketitle

% This should be about 250 words
\begin{abstract}
\textbf{Purpose} - This research paper aims to discover an optimal solution for parallel management and processing of financial markets data into  warehousing and analysis engines used for buy-sell decision-making. Methods analyzed herein are components of the Hadoop ecosystem. Included is an in-depth analysis of parallel processing - using the Elastic MapReduce package on Amazon Web Services - and data warehousing with Apache Hive. Apache NiFi is used to direct workflow automation for data migration into the warehouse. Finally, a complete assessment of the combinations of levels of the various Hadoop ecosystem applications used is provided in the context of statistical inference.
\\
\\
\textbf{Design, Methodology, and Approach} - One pertinent, underlying hypothesis within this study is to prove that there are differences in processing speeds between S3 and HDFS using normalized and optimized schema designs across multiple MapReduce configurations. This analysis is performed using 100 samples of the same volume of data in a repeated measures analysis using a Hotelling-T statistic. The two highest-performing configurations of S3 and HDFS are then assessed. (We need to get this information and report it) for the next section.
\\
\\
\textbf{Findings} - \textbf{EXAMPLE:} Applying S3 with an optimized schema using 10 reduces, map memory allocation of 2,048 mb and a reduce memory allocation of 4,096 mb is optimal for a more expensive S3 approach. Using HDFS from local storage with a configuration of 20 reduces, map memory allocation of 8,096 megabytes and reduce memory allocation of 10,020 megabytes is ideal for batch-level migrations and querying for large-volume processing. Across n repeated measures, using a one-tailed alpha, the resulting p-value is significant at `Pr > |t|` < x.xxxx (confidence interval (x1, x2)), indicating local storage from HDFS outperforms S3 when using the selected configurations. However, local data storage capacity does not scale well for HDFS compared to cloud-based S3.
\end{abstract}


% For peer review papers, you can put extra information on the cover
% page as needed:
% \ifCLASSOPTIONpeerreview
% \begin{center} \bfseries EDICS Category: 3-BBND \end{center}
% \fi
%
% For peerreview papers, this IEEEtran command inserts a page break and
% creates the second title. It will be ignored for other modes.
\IEEEpeerreviewmaketitle



\section{Introduction}
% The very first letter is a 2 line initial drop letter followed
% by the rest of the first word in caps.
% 
% form to use if the first word consists of a single letter:
% \IEEEPARstart{A}{demo} file is ....
% 
% form to use if you need the single drop letter followed by
% normal text (unknown if ever used by the IEEE):
% \IEEEPARstart{A}{}demo file is ....
% 
% Some journals put the first two words in caps:
% \IEEEPARstart{T}{his demo} file is ....
% 
% Here we have the typical use of a "T" for an initial drop letter
% and "HIS" in caps to complete the first word.
\IEEEPARstart{D}{ata} in the twenty-first century is expanding in volumes at exponential rates; every additional source of data that can act as a medium for data communication can obtain useful information, which, with modern technology, can be structured and stored, accessible to any who have the skills and need to make use of it. This information is increasingly profitable. However, with the increasing ability to capture and store data from many disparate sources, the need to store larger volumes of the data is likewise an increasing issue when it comes time to access and apply use, as many of the data gathered exist across very dynamic, diverse, and large partitions. As such, the scalability of storing and accessing big data must increase with it, relevant to the structures and locations of these data repositories. Developing a database in an ecosystem - Hadoop - that supports tools of two major concepts for achieving scalability within big data - data-parallelism and task-parallelism - our team has built a data warehousing solution to structure and store the data based on both optimized and normalized schema designs, drafted from entity-relationship models designed with the intention of enabling rapid storing and accessing through the Hadoop MapReduce process. Through a combination of these systems and data storage within Amazon Simple Storage Service (S3) and Apache Hadoop's native Hadoop Distributed File System (HDFS), we analyze performance between two approaches toward data- and task-parallelism using data gathered from the stock market.
\subsection{Apache Hadoop Ecosystem}
Motivated through the opportunity within big data and parallel computing, which enables massive amounts of data to be rapidly accessed for complex analysis and distributed across a scalable, cost-effective distribution of servers, we aligned our project with a modern database application, Hadoop, which provides a data lake ``ecosystem'' supporting both task- and data-parallelism across the many software applications within the Apache suite in addition to software that can be managed and accessed within a network of servers, called a cluster.
\subsection{Hadoop MapReduce}
The cluster of server nodes enables users to read data from the same source, simultaneously, as the data at that source is partitioned and processed across multiple servers – a master and at least one slave – through leveraging a processing framework called MapReduce to assign nodes for ``mapping'' and ``reducing'' by applying various configurations, such as related to memory allocation to and volumes of mappers and reducers. As data is mapped, it is reprocessed into a derivative data set, split into tuples that are then processed across the cluster of server nodes, in parallel, and reassembled in the reduce process from which it is delivered to the end-user, whether it be Enterprise Resource Planning (ERP) or a personal user running a SQL “SELECT” statement.
\subsection{Application Integration and Data Ingestion}
The parallelizability of Hadoop is central to this study as the primary objective is to rapidly store and access financial market data for buy-sell decision-making. The scope of applied analysis in this study is focused on the ability to scale and process a combination of quantitative stock market data and qualitative Twitter data related to the quantitative data. 
Quantitative data was gathered from a markets data vendor through an Application Programming Interface (API) on 15-minute intervals using R programming language. Qualitative data was gathered and processed through a Twitter API using Python programming language. The data, structured and stored into a Hive data warehouse is created and managed using Hive Query Language (HQL) stored both locally and within an Amazon S3 storage bucket. R will be used via ODBC to access and build proof-of-concept models using the data. Once developed, Python will be deployed within the data lake to process and derive data obtained through machine learning decisions.
\subsection{Big Data System Implementation}
In order to provide manageable storage repositories that scale well to size and data diversity, we have implemented our solution using S3 and HDFS. S3 and HDFS are designed for large sets of data. Therefore, integrity of data and ingestion systems are able to be well maintained. This is essential in an environment that may need to support many simultaneous users, each with different data needs. The HDFS implementation in this project is housed across three - one master and two slave - Amazon EC2 servers whereas S3 is a standalone repository within Amazon's cloud suite.
Additionally, Apache Hive is used as a data warehouse because it is operated with HQL, which is easily communicable for SQL users. Furthermore, Hive allows for storage of massive databases and tables and is well-suited for big data application integration, including the Apache software suite, of which Hadoop is a product. The data warehouse graphical user interface is provided by Cloudera Hue.
\subsection{Database Schema and Selection}
Flustered was the frivolous foe who quantified the schema not yet, ho. Finally, close the introduction out with some more explaining about normalized vs. optimized data, why that matters, and how we'll analyze it.
\subsection{Performance Metrics and Evaluation}
Performance is based on speed taken to process our 3 gigabytes of data - once processed into their respective storage systems - into the Hive data warehouse, which includes table creation and loading. We will use a repeated measures analysis and a one-tailed hypothesis test to determine optimal performance among S3 and HDFS groups, which is then used to compare and assess benchmarks between the combinations of MapReduce settings and database schemas among the best-performing S3 and HDFS configuration.




\section{Data}

This study investigates data warehouse models for housing stock price data and semi-structured alternative data. 
The stock price data was.
Twitter was chosen as the primary alternative data source because of ease-of-access the Twitter API and the large volume of available data.

\begin{table}
	\renewcommand{\arraystretch}{1.3}
	\caption{Data Features}
	\label{DataFeatures}
	\centering
	\begin{tabular}{c|l}
		\hline
		Source       & Features\\
		\hline
		\hline
		Stock Daily  & Prices: High, Low, Open, Close\\
		\hline
		\multirow{8}{*}{Stock Intraday} &  Prices: High, Low, Open, Close \\
		&  High Bollinger bands\\
		&  Mid Bollinger bands\\
		&  Low Bollinger bands\\ 
		&  Nominal moving average\\
		&  Historical moving average\\
		&  Signal moving average\\ 
		&  Exponential moving average\\
		&  Stochastic 5-day oscillators K, D\\
		\hline
		Tweets       & Text, URLs, Hashtags, Mentions, Users\\
		\hline
	\end{tabular}
\end{table}

\subsection{Stock Data}

The stock price data was collected through an API provided by Alpha Vantage.
This API provided access intraday stock prices, intraday price features, and daily prices.
Intraday stock data contained 35 features sampled at 15 minute intervals. 
The intraday features were from the following categories Bollinger bands, stochastic oscillators, moving averages, and exponential moving averages.
Approximately 1 millions rows of data were collected, which occurred from Oct. 04, 2019 to Oct. 24, 2019.
During the collection process, the data were recorded in files organized by day and category.
At the end of the collection process, the files for each category were combined and pushed to the HDFS datalake.


\subsection{Twitter Data}

sssss

\section{Data Warehouse Development}

This stock data is natively structured in a tabular form and naturally keyed by the combination of timestamp, stock symbol, and stock market.

\subsection{Normalized Schema}

%\begin{figure}
%	\centering
%	\includegraphics[width=2.5in]{NormalizedSchema.png}
%	\caption{Normalized Data Warehouse Schema}
%	\label{fig_sim}
%\end{figure}

Discuss normalizing the data - 3NF

\subsection{Optimized Schema}

Discuss the difference for optimization

\section{Technologies}

Rikel

\section{Results}



Maybe a table comparing read times for the different options

\begin{itemize}
	\item S3
	\item HDFS
	\item HDFS mapreduce settings?
\end{itemize}

\section{Analysis}

Analysis of the results


\section{Conclusion}

The conclusion goes here.

% Can use something like this to put references on a page
% by themselves when using endfloat and the captionsoff option.
\ifCLASSOPTIONcaptionsoff
  \newpage
\fi

\begin{thebibliography}{1}

\bibitem{BuildingtheDWCH11}
W. H. Inmon, "Unstructured Data and the Data Warehouse," in 
  \emph{Building the Data Warehouse},
  4th ed. Hoboken: Wiley, 2005, ch. x, sec. x.
  Accessed on Nov. 6, 2019 [Online]. 
  Available: \\ https://learning.oreilly.com/library/view/building-the-data/9780764599446

\bibitem{WarehouseDesignApproaches}
I. Moalla, A. Nabli, L. Bouzguendam and M. Hammami,
 "Data warehouse design approaches from social media: review and comparison,"
 Social Network Analysis and Mining., Vol. 7, no. 1, pp. 1-14, Jan. 2017.
 Accessed on: Nov. 6, 2019 [Online]. Available doi: 10.1007/s13278-017-0423-8

\end{thebibliography}


% that's all folks
\end{document}


